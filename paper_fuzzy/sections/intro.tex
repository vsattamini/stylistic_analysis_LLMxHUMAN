\noindent
Neste trabalho, exploramos o uso de lógica fuzzy como método de detecção de textos gerados por modelos de linguagem de grande porte (LLMs). Para isso, construímos um classificador fuzzy baseado em métricas estilométricas - propriedades da escrita que capturam padrões linguísticos, sintáticos e semânticos. Cada métrica é associada a uma função de pertinência que expressa o grau de pertencimento de um texto a variáveis linguísticas interpretáveis, como "alta fluência" ou "baixa variação lexical".

As funções de pertinência adotadas são triangulares, determinadas por três parâmetros $(a,b,c)$, amplamente utilizadas em sistemas fuzzy por sua simplicidade algorítmica e eficiência computacional~\cite{pedrycz1994}.

O interesse em utilizar lógica fuzzy na estilometria decorre da natureza intrinsecamente gradual da linguagem. Categorias como "texto bem estruturado" ou "escrita natural" dependem de critérios de pertinência. A lógica fuzzy ocupa um espaço entre empirismo e formalidade, aproximando-se da forma como utilizamos a linguagem natural para expressar incerteza e imprecisão~\cite{klir1995}. Essa característica a torna adequada para modelar a "gradualidade" no pertencimento de um texto a uma classe (autoral ou LLM).

Ao fuzificar métricas estilométricas e combiná-las no sistema de inferência fuzzy de regras "Se ... então", é possível estimar o grau de pertencimento de um texto a cada classe.

A principal vantagem da abordagem fuzzy é a \textbf{interpretabilidade}: ao contrário de modelos de caixa-preta, os graus de pertinência podem ser inspecionados e compreendidos por humanos, revelando em que medida cada dimensão estilométrica contribui para a decisão. Além disso, o sistema fuzzy permite incorporar conhecimento linguístico especializado na definição das funções de pertinência, embora aqui seja adotada uma abordagem orientada a dados (\textit{data-driven}), determinando os parâmetros a partir de quantis das distribuições observadas.

A lógica fuzzy tem sido amplamente aplicada em processamento de linguagem natural, especialmente em análise de sentimentos~\cite{vashishtha2023} e classificação de texto~\cite{liu2024}. Trabalhos recentes também exploram sistemas fuzzy interpretativos baseados em fundamentos axiomáticos~\cite{wang2024fuzzy}, demonstrando a viabilidade de sistemas transparentes e auditáveis. Contudo, até onde sabemos, nenhum estudo anterior aplicou lógica fuzzy especificamente à detecção de textos gerados por inteligência artificial. Enquanto LLMs têm sido analisados predominantemente por métodos estatísticos ou de aprendizado profundo, este trabalho propõe a utilização de sistemas de inferência fuzzy como alternativa explicável, eficiente e de fácil interpretação.

Os resultados apresentados demonstram que classificadores fuzzy simples podem alcançar desempenho competitivo (AUC de 89\%) em comparação com abordagens estatísticas e neurais mais complexas, preservando ao mesmo tempo transparência e interpretabilidade: características essenciais para aplicações em educação, moderação de conteúdo e integridade científica.