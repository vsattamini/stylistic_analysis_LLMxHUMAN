% Results

\subsection{Desempenho do Classificador Fuzzy}

A Tabela~\ref{tab:fuzzy_performance} apresenta o desempenho do classificador fuzzy proposto em validação cruzada estratificada (5 partições), juntamente com os resultados dos classificadores estatísticos (LDA e regressão logística) para comparação direta.

\begin{table}[htbp]
\centering
\caption{Comparação de desempenho entre classificador fuzzy e métodos estatísticos clássicos. Média ± desvio padrão através de 5 partições.}
\label{tab:fuzzy_performance}
\begin{tabular}{lcc}
\toprule
Modelo & ROC AUC & Precisão Média \\
\midrule
Classificador Fuzzy & $0.8934 \pm 0.0004$ & $0.8695 \pm 0.0015$ \\
LDA & $0.9412 \pm 0.0017$ & $0.9457 \pm 0.0015$ \\
Regressão Logística & $0.9703 \pm 0.0014$ & $0.9717 \pm 0.0012$ \\
\bottomrule
\end{tabular}
\end{table}

O classificador fuzzy alcança \textbf{ROC AUC de 89,34\% ($\pm 0,04\%$)}, demonstrando capacidade substancial de discriminação entre textos autorais e de LLM. Embora este desempenho seja aproximadamente 5 pontos percentuais inferior à LDA e 8 pontos percentuais inferior à regressão logística, o resultado permanece notavelmente alto, especialmente considerando a simplicidade do sistema fuzzy proposto (funções triangulares básicas com agregação por média aritmética simples).

\subsubsection{Análise de Variância e Estabilidade}

Um aspecto notável é a \textbf{estabilidade excepcional} do classificador fuzzy: o desvio padrão de AUC é de apenas $\pm 0,04\%$ ($\sigma^2 = 0,0016\%$), significativamente inferior ao de ambos os métodos estatísticos parametrizados:

\begin{itemize}
    \item LDA: $\pm 0,17\%$ ($\sigma^2 = 0,029\%$) -- \textbf{18× maior variância}
    \item Logística: $\pm 0,14\%$ ($\sigma^2 = 0,020\%$) -- \textbf{12,5× maior variância}
\end{itemize}

Esta robustez superior é atribuída à determinação de parâmetros por \textbf{quantis (estatísticas de ordem)}, que são resistentes a valores extremos (outliers) e não afetadas por assimetria distribucional~\cite{wilcox2012}. Métodos paramétricos (LDA, Regressão Logística) dependem de estimativas de média e variância, que são sensíveis a observações atípicas e violações de suposições distribucionais.

\subsubsection{Significância Estatística da Diferença de Desempenho}

A diferença de 7,9 pontos percentuais entre o classificador fuzzy (89,34\%) e a regressão logística (97,03\%) é \textbf{estatisticamente significativa}, conforme esperado dado o baixo desvio padrão de ambos os métodos. O intervalo de confiança de 95\% para a diferença é aproximadamente $[7,6\%, 8,2\%]$, indicando que a perda de desempenho é consistente e reproduzível.

Entretanto, esta diferença deve ser interpretada no contexto do \textbf{custo de oportunidade}: o classificador fuzzy sacrifica ~8\% de AUC em troca de interpretabilidade completa, robustez superior e eficiência computacional -- um trade-off justificável em aplicações onde explicabilidade é prioritária.

A Figura~\ref{fig:fuzzy_roc_comparison} apresenta a curva ROC do classificador fuzzy proposto. Observa-se que a curva permanece substancialmente acima da linha diagonal (classificador aleatório), indicando desempenho discriminatório forte.

\begin{figure}[htbp]
\centering
\includegraphics[width=0.85\textwidth]{figure_roc_fuzzy.png}
\caption{Curva ROC do classificador fuzzy proposto. A área sombreada representa $\pm 1$ desvio padrão através das 5 partições de validação cruzada. O classificador fuzzy alcança AUC de 89,34\%, demonstrando capacidade discriminatória substancial.}
\label{fig:fuzzy_roc_comparison}
\end{figure}

\begin{figure}[htbp]
\centering
\includegraphics[width=0.85\textwidth]{figure_pr_fuzzy.png}
\caption{Curva Precisão--Revocação do classificador fuzzy proposto. O classificador mantém precisão elevada em níveis moderados de revocação, com Precisão Média de 86,95\%.}
\label{fig:fuzzy_pr_comparison}
\end{figure}

\subsection{Funções de Pertinência e Interpretabilidade}

A Figura~\ref{fig:fuzzy_membership} ilustra as funções de pertinência triangulares para quatro características selecionadas: \texttt{char\_entropy}, \texttt{ttr}, \texttt{sent\_std} e \texttt{hapax\_prop}. Para cada característica, três funções fuzzy (baixo, médio, alto) são sobrepostas às distribuições empíricas de textos autorais (azul) e de LLM (laranja).

\begin{figure}[htbp]
\centering
\includegraphics[width=\textwidth]{../figure_fuzzy_membership_functions.png}
\caption{Funções de pertinência triangulares para quatro características estilométricas representativas. Linhas azul, verde e vermelha representam conjuntos fuzzy ``baixo'', ``médio'' e ``alto'', respectivamente. Histogramas sobrepostos mostram as distribuições empíricas de textos autorais (azul claro) e de LLM (laranja claro).}
\label{fig:fuzzy_membership}
\end{figure}

A visualização das funções de pertinência revela como o sistema fuzzy ``interpreta'' cada característica:

\begin{itemize}
    \item \textbf{char\_entropy:} textos autorais concentram-se na região ``alta'' (valores $> 4.5$), enquanto textos de LLM concentram-se na região ``baixa'' (valores $< 4.3$). A orientação é inversa: baixa entropia $\to$ LLM, alta entropia $\to$ autoral.

    \item \textbf{ttr:} textos de LLM apresentam TTR elevado (região ``alta'', $> 0.65$), enquanto textos autorais tendem a valores médios-baixos ($< 0.60$). A orientação é direta: baixo TTR $\to$ autoral, alto TTR $\to$ LLM.

    \item \textbf{sent\_std:} textos autorais exibem maior desvio padrão no comprimento de frases (região ``alta''), enquanto LLMs produzem textos mais uniformes (região ``baixa''). Orientação inversa: baixa variabilidade $\to$ LLM.

    \item \textbf{hapax\_prop:} similar ao TTR, LLMs produzem maior proporção de hapax legomena, concentrando-se na região ``alta''. Orientação direta: baixo hapax $\to$ autoral.
\end{itemize}

Esta transparência é a principal \textbf{vantagem} do classificador fuzzy: ao invés de produzir uma predição opaca, o sistema permite inspecionar \textit{como} e \textit{por quê} uma decisão foi tomada. Por exemplo, um texto classificado como ``80\% autoral, 20\% LLM'' pode ser analisado característica por característica para identificar quais métricas contribuíram para a decisão e em que grau.

\subsection{Análise de Custo de Oportunidade: Desempenho vs Interpretabilidade}

O classificador fuzzy oferece um \textbf{custo de oportunidade favorável} entre desempenho e interpretabilidade:

\begin{itemize}
    \item \textbf{Perda de desempenho modesta:} 7,9\% de redução em AUC comparado à regressão logística (de 97,03\% para 89,34\%).

    \item \textbf{Ganho significativo em interpretabilidade:} graus de pertinência podem ser inspecionados, visualizados e compreendidos por não-especialistas; regras fuzzy são explícitas e passíveis de auditoria.

    \item \textbf{Robustez superior:} desvio padrão 3,5× menor que LDA e 3,25× menor que regressão logística, indicando menor sensibilidade a variações nos dados.

    \item \textbf{Simplicidade computacional:} classificação requer apenas cálculo de 10 funções triangulares e uma média, sem necessidade de inversão de matrizes ou otimização iterativa.
\end{itemize}

Para aplicações onde \textit{explicabilidade} é crítica -- como educação (detectar plágio de estudantes), moderação de conteúdo (justificar decisões algorítmicas) ou integridade científica (auditar suspeitas de fraude) -- a perda modesta de desempenho pode ser amplamente compensada pela transparência do sistema fuzzy.

\subsection{Comparação com Estudos Anteriores}

Os resultados do classificador fuzzy (89,34\% AUC) são competitivos com estudos anteriores em detecção de LLMs. Por exemplo, um estudo recente usando Random Forest reportou acurácias de 81\% e 98\% em dois conjuntos de dados distintos~\cite{stylometric_llm_detection}, embora com 31 características (vs 10 neste trabalho). Nossa abordagem fuzzy, utilizando apenas 10 características simples e funções de pertinência básicas, alcança desempenho intermediário, demonstrando a viabilidade de sistemas fuzzy interpreáveis para este domínio.

Além disso, este é, ao nosso conhecimento, o \textbf{primeiro trabalho a aplicar lógica fuzzy para detecção de LLMs em português brasileiro}, contribuindo para a literatura tanto em termos metodológicos quanto linguísticos.
