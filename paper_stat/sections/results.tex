% Results

\subsection{Comparação Estatística das Características}

A Tabela~\ref{tab:statistical_tests} apresenta os resultados dos testes U de Mann--Whitney para todas as 10 características estilométricas. Nove das dez características mostram diferenças altamente significativas ($p < 0.001$) entre textos autorais e de LLM, mantendo-se significativas após correção FDR ($q < 0.001$). A única exceção é \texttt{fk\_grade}, que retornou valores zero para ambas as classes por ser uma métrica específica para inglês, resultando em $p = 1.000$ como esperado.

Os tamanhos de efeito, medidos pelo delta de Cliff, revelam que \textbf{cinco características} apresentam efeitos \textbf{grandes} ($|\delta| \geq 0.474$): \texttt{char\_entropy} ($\delta = -0.881$), \texttt{sent\_std} ($\delta = -0.790$), \texttt{sent\_burst} ($\delta = -0.663$), \texttt{ttr} ($\delta = 0.616$) e \texttt{hapax\_prop} ($\delta = 0.564$). Três características apresentam efeitos \textbf{médios}: \texttt{herdan\_c} ($\delta = 0.450$), \texttt{bigram\_repeat\_ratio} ($\delta = -0.424$) e \texttt{func\_word\_ratio} ($\delta = 0.378$). Apenas \texttt{first\_person\_ratio} ($\delta = -0.049$) apresenta efeito negligenciável.

\begin{table}[htbp]
\centering
\caption{Resultados dos testes U de Mann--Whitney comparando características estilométricas entre textos autorais e de LLM. Valores-$q$ ajustados por FDR (Benjamini--Hochberg). H = autoral.}
\label{tab:statistical_tests}
\small
\begin{tabular}{lrrrrc}
\toprule
Característica & Mediana (H) & Mediana (LLM) & $p$-valor & Delta de Cliff & Efeito \\
\midrule
\texttt{sent\_mean} & 20.000 & 16.500 & $<0.001$ & $-0.290$ & Pequeno \\
\texttt{sent\_std} & 12.487 & 4.528 & $<0.001$ & $-0.790$ & Grande \\
\texttt{sent\_burst} & 0.640 & 0.319 & $<0.001$ & $-0.663$ & Grande \\
\texttt{ttr} & 0.570 & 0.735 & $<0.001$ & $+0.616$ & Grande \\
\texttt{herdan\_c} & 0.903 & 0.929 & $<0.001$ & $+0.450$ & Médio \\
\texttt{hapax\_prop} & 0.417 & 0.581 & $<0.001$ & $+0.564$ & Grande \\
\texttt{char\_entropy} & 4.560 & 4.254 & $<0.001$ & $-0.881$ & Grande \\
\texttt{func\_word\_ratio} & 0.312 & 0.347 & $<0.001$ & $+0.378$ & Médio \\
\texttt{first\_person\_ratio} & 0.002 & 0.000 & $1.6 \times 10^{-47}$ & $-0.049$ & Negligível \\
\texttt{bigram\_repeat\_ratio} & 0.066 & 0.030 & $<0.001$ & $-0.424$ & Médio \\
\bottomrule
\end{tabular}
\end{table}

\subsection{Interpretação das Características Discriminantes}

As características mais discriminantes revelam padrões consistentes:

\noindent\textbf{Textos humanos são caracterizados por:}
\begin{itemize}
    \item \textbf{Maior diversidade em nível de caractere:} a entropia de caracteres ($\delta = -0.881$) é substancialmente maior, indicando distribuições de caracteres mais heterogêneas.
    \item \textbf{Maior variabilidade estrutural:} desvio padrão do comprimento de frases ($\delta = -0.790$) e burstiness ($\delta = -0.663$) são ambos elevados, refletindo estruturas sintáticas mais irregulares.
    \item \textbf{Maior repetição de bigramas:} textos autorais tendem a repetir combinações de palavras com maior frequência ($\delta = -0.424$).
\end{itemize}

\noindent\textbf{Textos de LLM são caracterizados por:}
\begin{itemize}
    \item \textbf{Maior diversidade lexical:} TTR ($\delta = +0.616$) e proporção de hapax ($\delta = +0.564$) elevados indicam vocabulário menos repetitivo, possivelmente devido ao treinamento em corpora extremamente diversos.
    \item \textbf{Maior uso de palavras funcionais:} proporção de palavras funcionais ($\delta = +0.378$) ligeiramente superior, sugerindo estilo mais formal ou explícito.
    \item \textbf{Maior uniformidade estrutural:} menor variação no comprimento de frases, gerando textos mais ``regulares''.
\end{itemize}

A Figura~\ref{fig:boxplots} apresenta diagramas de caixa para todas as características, ilustrando graficamente essas diferenças. As medianas, quartis e valores atípicos confirmam visualmente a separação entre as distribuições.

\begin{figure}[htbp]
\centering
\includegraphics[width=\textwidth]{../figure_boxplots.png}
\caption{Diagramas de caixa comparando as distribuições de características estilométricas entre textos autorais (azul) e de LLM (vermelho). Asteriscos indicam significância estatística: *** = $p < 0.001$.}
\label{fig:boxplots}
\end{figure}

\subsection{Análise de Componentes Principais}

A análise de componentes principais revela que os dois primeiros componentes (PC1 e PC2) explicam cumulativamente \textbf{54,15\% da variância} dos dados: PC1 explica 38,11\% e PC2 explica 16,03\%. A Figura~\ref{fig:pca} mostra o gráfico de dispersão no espaço PC1--PC2, onde se observa \textbf{separação clara} entre as duas classes, embora com alguma sobreposição.

\begin{figure}[htbp]
\centering
\includegraphics[width=0.8\textwidth]{../pca_scatter.png}
\caption{Gráfico de dispersão dos dois primeiros componentes principais (PC1 vs PC2). Textos humanos (azul) concentram-se em PC1 negativo e PC2 positivo; textos de LLM (vermelho) em PC1 positivo e PC2 negativo.}
\label{fig:pca}
\end{figure}

As cargas fatoriais de PC1 indicam que este componente representa um eixo de tipicidade de LLM: características como TTR, hapax e C de Herdan têm pesos positivos (favorecem LLM), enquanto coeficiente de variação, desvio padrão de frases e entropia de caracteres têm pesos negativos (favorecem textos autorais). PC2 representa primariamente variabilidade estrutural (coeficiente de variação e desvio padrão têm pesos positivos altos).

A Figura~\ref{fig:correlation} apresenta a matriz de correlação entre as características. Observa-se forte correlação positiva entre TTR, hapax e C de Herdan ($r > 0.7$), formando um agrupamento de diversidade lexical. O desvio padrão de frases e o coeficiente de variação também são fortemente correlacionados ($r = 0.72$), como esperado pela definição do coeficiente de variação ($\sigma/\mu$).

\begin{figure}[htbp]
\centering
\includegraphics[width=0.85\textwidth]{../figure_correlation_heatmap.png}
\caption{Matriz de correlação de Pearson entre as características estilométricas. Cores quentes (vermelho) indicam correlação positiva; cores frias (azul) indicam correlação negativa.}
\label{fig:correlation}
\end{figure}

\subsection{Desempenho dos Classificadores}

A Tabela~\ref{tab:classification} resume o desempenho dos dois classificadores lineares em validação cruzada estratificada (5 partições). Ambos os modelos alcançam desempenho excelente, com \textbf{regressão logística superando LDA} em aproximadamente 3 pontos percentuais.

\begin{table}[htbp]
\centering
\caption{Desempenho dos classificadores em validação cruzada (5 partições). Média ± desvio padrão.}
\label{tab:classification}
\begin{tabular}{lcc}
\toprule
Modelo & ROC AUC & Precisão Média \\
\midrule
LDA & $0.9412 \pm 0.0017$ & $0.9457 \pm 0.0015$ \\
Regressão Logística & $0.9703 \pm 0.0014$ & $0.9717 \pm 0.0012$ \\
\bottomrule
\end{tabular}
\end{table}

A regressão logística atinge \textbf{ROC AUC de 97,03\%}, demonstrando capacidade quase perfeita de distinguir textos autorais de textos de LLM. O desvio padrão extremamente baixo ($\pm 0.14\%$) indica alta estabilidade do modelo através das partições. A LDA, embora ligeiramente inferior, ainda alcança excelente desempenho (94,12\% AUC), confirmando que a separação linear é suficiente para este problema.

\subsection{Validação Estatística dos Modelos Multivariados}

\subsubsection{Análise Discriminante Linear}

A Tabela \ref{tab:lda_anova} apresenta os resultados do teste de Lambda de Wilks para a LDA:

\begin{table}[htbp]
\centering
\caption{Validação estatística da Análise Discriminante Linear}
\label{tab:lda_anova}
\begin{tabular}{lcccc}
\hline
\textbf{Estatística} & \textbf{Valor} & \textbf{F} & \textbf{gl} & \textbf{$p$-valor} \\
\hline
Lambda de Wilks ($\Lambda$) & [VALOR] & [VALOR] & [gl1, gl2] & $< 0.001$ \\
\hline
\end{tabular}
\end{table}

O valor de Lambda de Wilks = [VALOR] indica [interpretação]. A estatística $F$ = [VALOR] com $p < 0.001$ rejeita fortemente a hipótese nula de igualdade de centroides, confirmando que a LDA discrimina significativamente entre textos humanos e LLM.

\subsubsection{Regressão Logística}

A Tabela \ref{tab:logit_validation} apresenta as medidas de validação do modelo logístico:

\begin{table}[htbp]
\centering
\caption{Validação estatística da Regressão Logística}
\label{tab:logit_validation}
\begin{tabular}{lcc}
\hline
\textbf{Medida} & \textbf{Valor} & \textbf{Interpretação} \\
\hline
Razão de verossimilhança ($G$) & [VALOR] & $p < 0.001$ \\
Hosmer-Lemeshow ($H$) & [VALOR] & $p = [VALOR]$ \\
Deviance & [VALOR] & - \\
Pseudo-$R^2$ (McFadden) & [VALOR] & Excelente ajuste \\
\hline
\end{tabular}
\end{table}

O teste de razão de verossimilhança ($G$ = [VALOR], $p < 0.001$) indica que o modelo completo é significativamente melhor que o modelo nulo. O teste de Hosmer-Lemeshow ($H$ = [VALOR], $p$ = [VALOR]) [não rejeita / rejeita] a hipótese de bom ajuste. O pseudo-$R^2$ de McFadden = [VALOR] indica [interpretação].

As Figuras~\ref{fig:roc} e~\ref{fig:pr} apresentam as curvas ROC e Precisão--Revocação, respectivamente, agregadas através das 5 partições. As bandas de confiança ($\pm 1$ desvio padrão) são estreitas, refletindo a consistência dos resultados.

\begin{figure}[htbp]
\centering
\includegraphics[width=0.85\textwidth]{figure_roc_stat.png}
\caption{Curvas ROC para LDA e regressão logística. Linhas sólidas representam a média das 5 partições; áreas sombreadas indicam $\pm 1$ desvio padrão. A linha tracejada representa o classificador aleatório (AUC = 0.50).}
\label{fig:roc}
\end{figure}

\begin{figure}[htbp]
\centering
\includegraphics[width=0.85\textwidth]{figure_pr_stat.png}
\caption{Curvas Precisão--Revocação para LDA e regressão logística. Ambos os modelos mantêm alta precisão mesmo em altos níveis de revocação, indicando baixas taxas de falsos positivos e falsos negativos.}
\label{fig:pr}
\end{figure}

Os resultados demonstram que métodos estatísticos clássicos são altamente eficazes para distinguir textos autorais de textos gerados por LLMs em português do Brasil, confirmando achados anteriores em língua inglesa e estendendo-os para outro idioma e contexto.
