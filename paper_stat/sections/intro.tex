% Introduction

\noindent
A emergência de modelos de linguagem de grande porte (LLMs) criou preocupações quanto à detecção de conteúdo gerado automaticamente. A detecção de autoria computacional tem raízes históricas sólidas, iniciando com o trabalho seminal de Mosteller e Wallace~\cite{mosteller1964} sobre os artigos Federalistas e posteriormente formalizada por Burrows~\cite{burrows2002} com a medida Delta para diferenciação estilística. Trabalhos recentes demonstram que essas técnicas estilométricas clássicas permanecem eficazes para distinguir textos autorais de textos gerados por LLMs~\cite{stamatatos2009,huang2024}.

Estudos em múltiplos idiomas confirmam a viabilidade da abordagem estilométrica: Herbold et al.~\cite{stylometric_llm_detection} reportaram 81--98\% de acurácia usando 31 características e Random Forest; Zaitsu e Jin~\cite{zaitsu2023} alcançaram 100\% de precisão em textos japoneses; Przystalski et al.~\cite{przystalski2025} demonstraram que estilometria reconhece LLMs mesmo em pequenas amostras (0,87--0,98 de acurácia); e Berriche e Larabi-Marie-Sainte~\cite{berriche2024} atingiram 100\% usando 33 características estilométricas com XGBoost. Esses resultados evidenciam que características como comprimento médio de frases, relação tipo-token, entropia de caracteres~\cite{shannon1948}, proporção de palavras funcionais~\cite{stamatatos2009} e burstiness~\cite{madsen2005} contêm sinais fortes sobre a origem do texto.

Este estudo contribui para a literatura de detecção de LLMs ao fornecer uma primeira análise estilométrica para detecção de textos gerados por LLMs em português do Brasil. Não foi encontrado aplicação de análise estilométrica a textos de LLM em português. Utilizou-se um conjunto de dados balanceado com mais de 1,2 milhões de amostras de múltiplas fontes (BrWaC~\cite{brwac}, ShareGPT-Portuguese~\cite{sharegpt_portuguese}, Canarim~\cite{canarim}).\footnote{Desde a compilação deste corpus, novos recursos em português surgiram, incluindo GigaVerbo com 200B tokens~\cite{correa2024} e PTT5-v2~\cite{piau2024}, que podem beneficiar trabalhos futuros.}
