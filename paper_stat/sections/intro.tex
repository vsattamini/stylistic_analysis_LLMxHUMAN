% Introduction

\noindent
A emergência de modelos de linguagem de grande porte (LLMs) criou preocupações quanto à detecção de conteúdo gerado automaticamente. Trabalhos recentes mostram que é possível distinguir textos humanos de textos de LLM usando métricas estilométricas clássicas e classificadores relativamente simples. Por exemplo, um estudo usando 31 características estilométricas e um classificador Random Forest reportou acurácias de 81 % e 98 % em dois conjuntos de dados distintos, superando métodos de referência
arxiv.org
. Esse resultado evidencia que características como comprimento médio de frases, relação tipo‑token, entropia de caracteres, proporção de palavras funcionais e repetição de n‑gramas contêm sinais fortes sobre a origem do texto. Além disso, o mesmo estudo contribuiu com 12 novas métricas estilométricas e demonstrou a importância de avaliações multi‑domínio
arxiv.org
.

No contexto deste trabalho, propomos um pipeline estatístico para caracterizar e comparar textos humanos e de LLM a partir de um único corpus bilíngue. A abordagem utiliza métodos descritivos, testes não paramétricos (Mann–Whitney, Kruskal–Wallis) e modelos multivariados (PCA, análise discriminante linear e regressão logística). As métricas estilométricas extraídas também servirão de base para o segundo artigo, permitindo um uso eficiente do tempo. A motivação é demonstrar que técnicas estatísticas clássicas, alinhadas às referências recomendadas pela disciplina (Morrison, Siegel & Castellan, Mood & Graybill e Bussab & Morettin), são suficientes para revelar diferenças significativas entre textos humanos e gerados por IA. Espera‑se que os resultados confirmem achados anteriores, ao mesmo tempo em que possibilitem análises adicionais, como tamanho de efeito e intervalos de confiança, ausentes em alguns trabalhos da literatura.


A emergência de modelos de linguagem de grande porte (LLMs) criou preocupações quanto à detecção de conteúdo gerado automaticamente. Trabalhos recentes mostram que é possível distinguir textos humanos de textos de LLM usando métricas estilométricas clássicas e classificadores relativamente simples. Por exemplo, um estudo usando 31 características estilométricas e um classificador Random Forest reportou acurácias de 81 % e 98 % em dois conjuntos de dados distintos, superando métodos de referência
arxiv.org
. Esse resultado evidencia que características como comprimento médio de frases, relação tipo‑token, entropia de caracteres, proporção de palavras funcionais e repetição de n‑gramas contêm sinais fortes sobre a origem do texto. Além disso, o mesmo estudo contribuiu com 12 novas métricas estilométricas e demonstrou a importância de avaliações multi‑domínio
arxiv.org
.

No contexto deste trabalho, propomos um pipeline estatístico para caracterizar e comparar textos humanos e de LLM a partir de um único corpus bilíngue. A abordagem utiliza métodos descritivos, testes não paramétricos (Mann–Whitney, Kruskal–Wallis) e modelos multivariados (PCA, análise discriminante linear e regressão logística). As métricas estilométricas extraídas também servirão de base para o segundo artigo, permitindo um uso eficiente do tempo. A motivação é demonstrar que técnicas estatísticas clássicas, alinhadas às referências recomendadas pela disciplina (Morrison, Siegel & Castellan, Mood & Graybill e Bussab & Morettin), são suficientes para revelar diferenças significativas entre textos humanos e gerados por IA. Espera‑se que os resultados confirmem achados anteriores, ao mesmo tempo em que possibilitem análises adicionais, como tamanho de efeito e intervalos de confiança, ausentes em alguns trabalhos da literatura.


