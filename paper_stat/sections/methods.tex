% Methods

\subsection{Conjunto de Dados}

Utilizou-se um corpus balanceado de textos em português do Brasil contendo 100.000 amostras (50.000 autorais, 50.000 de LLMs), extraídas por amostragem estratificada de um conjunto maior com 2.331.317 documentos originais provenientes de 5 fontes distintas. As fontes de texto autoral incluem: (i) BrWaC (Brazilian Web as Corpus)~\cite{brwac}, um grande corpus web de textos brasileiros; e (ii) BoolQ~\cite{boolq}, contendo passagens de contexto para perguntas booleanas. As fontes de texto gerado por LLM incluem: (i) ShareGPT-Portuguese~\cite{sharegpt_portuguese}, conversas em português extraídas da plataforma ShareGPT; (ii) resenhas do IMDB traduzidas para português por modelos de tradução automática (classificadas como texto LLM); e (iii) o dataset Canarim~\cite{canarim}, contendo saídas geradas por LLMs.

Os textos foram previamente filtrados por comprimento mínimo de 100 caracteres e máximo de 10.000 caracteres, sendo textos muito longos segmentados em chunks de até 10.000 caracteres sem sobreposição. A segmentação priorizou quebras naturais de texto (pontos finais, parágrafos e espaços). O balanceamento foi obtido por subamostragem (downsampling) da classe majoritária e sobreamostragem (upsampling) da classe minoritária, resultando em proporções exatamente iguais (50\%/50\%). A amostra de 100.000 documentos foi selecionada aleatoriamente com semente fixa (\texttt{seed=42}) para reprodutibilidade.

Para prevenir vazamento de dados (data leakage), verificamos que os textos não apresentam agrupamentos estruturais por autor, tópico ou sessão de geração. A validação cruzada estratificada mantém o balanço de classes entre os folds, garantindo amostras independentes em conjuntos de treino e teste. Esta abordagem evita viés de avaliação documentado em estudos anteriores~\cite{kohavi1995}.

\subsection{Extração de Características Estilométricas}

Cada amostra de texto foi processada pelo módulo \texttt{src/features.py}, que calcula 10 métricas estilométricas contínuas em português. As características foram selecionadas por capturarem aspectos complementares da estrutura estatística e lexical dos textos:

\begin{enumerate}
    \item \textbf{Comprimento médio de frase} (\texttt{sent\_mean}): medida de tendência central do número de palavras por frase, calculada após segmentação em pontos finais, exclamações e interrogações. Representa o padrão sintático dominante no texto.

    \item \textbf{Desvio padrão do comprimento de frase} (\texttt{sent\_std}): medida de dispersão que quantifica a variabilidade na estrutura sintática. Desvio padrão populacional (ddof=0) do número de palavras por frase. Textos com alta variabilidade sintática apresentam maior desvio padrão.

    \item \textbf{Burstiness} (\texttt{sent\_burst}): coeficiente de variação do comprimento de frase, calculado como $\text{burstiness} = \sigma / \mu$, onde $\sigma$ é o desvio padrão e $\mu$ a média do comprimento de frases. Esta medida de variabilidade relativa é útil para discriminação pois normaliza a dispersão pela tendência central, permitindo comparações entre textos com diferentes médias de comprimento.

    \item \textbf{Relação tipo-token} (\texttt{ttr}): medida de diversidade lexical calculada como $\text{TTR} = V/N$, onde $V$ é o número de tipos (palavras únicas) e $N$ o número de tokens (total de palavras). Quantifica a dispersão do vocabulário: valores altos indicam maior variedade lexical.

    \item \textbf{C de Herdan} (\texttt{herdan\_c}): medida de diversidade lexical normalizada para o comprimento do texto, calculada como $C = \log(V) / \log(N)$~\cite{herdan1960}. Corrige o viés de comprimento presente no TTR, fornecendo estimativa mais estável da dispersão vocabular.

    \item \textbf{Proporção de hapax legomena} (\texttt{hapax\_prop}): fração de palavras que ocorrem exatamente uma vez no texto, relativa ao total de tokens. Captura a raridade lexical e está relacionada à riqueza vocabular.

    \item \textbf{Entropia de caracteres} (\texttt{char\_entropy}): medida de variabilidade na distribuição de caracteres, calculada como $H = -\sum p(c) \log_2 p(c)$, onde $p(c)$ é a probabilidade de cada caractere. Alta entropia indica distribuição mais uniforme (maior incerteza); baixa entropia indica concentração em poucos caracteres.

    \item \textbf{Proporção de palavras funcionais} (\texttt{func\_word\_ratio}): fração de tokens pertencentes a uma lista curada de 86 palavras funcionais do português (artigos, preposições, conjunções e pronomes). Palavras funcionais são estáveis entre textos e apresentam menor variância que palavras de conteúdo, sendo discriminativas para detecção de autoria.

    \item \textbf{Proporção de pronomes de primeira pessoa} (\texttt{first\_person\_ratio}): fração de tokens que são pronomes de primeira pessoa (eu, me, mim, meu, minha, nós, nos, nosso, nossa). Captura o grau de subjetividade e perspectiva pessoal no texto.

    \item \textbf{Proporção de bigramas repetidos} (\texttt{bigram\_repeat\_ratio}): fração de bigramas únicos que aparecem mais de uma vez no texto. Mede a redundância de estruturas locais de dois tokens.
\end{enumerate}

Todas as métricas foram calculadas após tokenização baseada em expressões regulares (padrão \texttt{\textbackslash b\textbackslash w+\textbackslash b}). Do ponto de vista estatístico, todas as características são \textbf{variáveis contínuas} (admitem valores fracionários e permitem operações aritméticas). Quanto à escala de medida, nove características estão em \textbf{escala de razão} (possuem zero absoluto e razões interpretáveis), enquanto a entropia de caracteres está em \textbf{escala de intervalo} (diferenças são interpretáveis mas razões não têm significado, pois não há zero absoluto natural).

\subsection{Testes Estatísticos Não Paramétricos}

Para cada característica, comparamos as distribuições entre textos autorais e de LLM usando o teste U de Mann--Whitney~\cite{mann1947}, um teste não paramétrico para duas amostras independentes. Testes não paramétricos foram escolhidos porque as características estilométricas frequentemente violam pressupostos de testes paramétricos: (i) as distribuições não são normais (verificado por inspeção visual e testes de normalidade), (ii) há presença de outliers em dados linguísticos, e (iii) algumas características apresentam assimetria acentuada. O teste de Mann--Whitney compara as medianas das distribuições sem assumir forma distribucional específica. Calculamos valores-$p$ bicaudais para cada teste.

O tamanho de efeito foi quantificado pelo delta de Cliff ($\delta$)~\cite{cliff1993, cliff1996}, calculado como $\delta = (D^+ - D^-) / (n_1 \times n_2)$, onde $D^+$ é o número de pares em que valores do grupo autoral excedem valores do grupo LLM, $D^-$ é o número de pares na relação inversa, e $n_1$, $n_2$ são os tamanhos amostrais. O delta de Cliff admite interpretação probabilística: representa a diferença entre $P(X_{\text{autoral}} > X_{\text{LLM}})$ e $P(X_{\text{autoral}} < X_{\text{LLM}})$, estimando o grau de dominância estocástica de uma distribuição sobre a outra. O delta varia entre $-1$ e $+1$, onde valores próximos de zero indicam sobreposição completa entre as distribuições. Seguindo Romano et al.~\cite{romano2006}, interpretamos $|\delta| < 0.147$ como efeito negligenciável, $0.147 \leq |\delta| < 0.330$ como pequeno, $0.330 \leq |\delta| < 0.474$ como médio e $|\delta| \geq 0.474$ como grande.

Dado que realizamos testes múltiplos (10 características), aplicamos a correção de Benjamini--Hochberg~\cite{benjamini1995} para controlar a taxa de falsas descobertas (FDR). Esta correção fornece valores-$q$ ajustados, mantendo o controle sobre a proporção esperada de falsas rejeições entre todas as rejeições.

\subsection{Análise de Componentes Principais (PCA)}

Para visualizar a estrutura multivariada dos dados, aplicamos análise de componentes principais~\cite{jolliffe2002} às 10 características estilométricas. As variáveis foram previamente padronizadas (média zero, desvio padrão unitário) usando \texttt{StandardScaler} do scikit-learn~\cite{scikit-learn}. Retemos os dois primeiros componentes principais (PC1 e PC2) para visualização bidimensional. Reportamos a proporção de variância explicada por cada componente e os loadings (pesos) de cada característica original sobre os componentes.

\subsection{Modelos de Classificação}

Avaliamos três modelos para classificação binária:

\begin{enumerate}
    \item \textbf{Análise Discriminante Linear (LDA):} um classificador generativo que assume distribuições Gaussianas multivariadas para cada classe com matrizes de covariância iguais, buscando a direção de projeção $w = S_W^{-1}(\mu_1 - \mu_2)$ que maximiza a separação entre classes~\cite{fisher1936, mclachlan2004}.

    \item \textbf{Regressão Logística:} um modelo discriminativo que estima diretamente a probabilidade posterior através da função logística $P(Y=1|X) = 1 / (1 + \exp(-(\beta_0 + \sum \beta_i x_i)))$, sem assumir normalidade das features~\cite{hosmer2013}.

    \item \textbf{Classificador Fuzzy:} um sistema baseado em regras com funções de pertinência triangulares data-driven (definidas por quantis 33\%, 50\%, 66\%), agregação por média aritmética e inferência tipo Takagi-Sugeno ordem-zero. Detalhes completos em trabalho complementar sobre classificação fuzzy.
\end{enumerate}

Os modelos LDA e Regressão Logística foram treinados sobre as 10 características padronizadas (média zero, desvio padrão unitário). Para a regressão logística, utilizamos \texttt{max\_iter=1000} e sem regularização. O classificador fuzzy opera diretamente sobre as features não-padronizadas.

\subsection{Validação Cruzada e Métricas de Desempenho}

Empregamos validação cruzada estratificada com 5 folds (\texttt{StratifiedKFold}, \texttt{random\_state=42})~\cite{kohavi1995} para avaliar o desempenho dos classificadores. A estratificação garante que cada fold mantenha a proporção 50/50 de classes. Cada fold utiliza 80\% dos dados para treino (4 folds) e 20\% para teste (1 fold).

A métrica primária de avaliação é a \textbf{área sob a curva ROC (AUC)}~\cite{fawcett2006}, que resume a capacidade do modelo de discriminar entre as classes em todos os limiares de decisão. A curva ROC plota a taxa de verdadeiros positivos (TPR) versus a taxa de falsos positivos (FPR) para diferentes thresholds. O AUC possui interpretação probabilística: a probabilidade de que um texto LLM aleatório receba score maior que um texto autoral aleatório.

Reportamos a média e o desvio padrão de AUC ao longo dos 5 folds. Todas as análises foram implementadas em Python 3 utilizando as bibliotecas pandas~\cite{pandas}, NumPy~\cite{numpy}, scikit-learn~\cite{scikit-learn} e SciPy~\cite{scipy} para testes estatísticos.
