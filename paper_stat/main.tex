\documentclass[11pt]{article}
\usepackage[margin=1in]{geometry}
\usepackage[brazilian]{babel}
\usepackage{amsmath,amssymb}
\usepackage{graphicx}
\usepackage{booktabs}
\usepackage[alf,abnt-emphasize=bf,abnt-repeated-title-ignore=yes,abnt-thesis-year=final]{abntex2cite}
\makeatletter
\@ifpackageloaded{hyperref}{%
  \usepackage[hyperfootnotes=false]{hyperref}%
}{%
  \usepackage[hyperfootnotes=false]{hyperref}%
}
\makeatother
\hypersetup{colorlinks=true,linkcolor=blue,citecolor=blue,urlcolor=blue}

% Redefinir nome da seção de referências
\renewcommand{\refname}{Referências}

\title{Análise Estilométrica de Textos Humanos e de LLMs Usando Métodos Estatísticos}
\author{Victor Löfgren Sattamini\\Programa de Pós-Graduação em Ciências Computacionais e Modelagem Matemática (PPG-CompMat)\\IME UERJ}
\date{10 de Novembro de 2025}

\begin{document}

% Redefinir maketitle para garantir centralização
\makeatletter
\renewcommand{\@maketitle}{%
  \newpage
  \null
  \vskip 2em%
  \begin{center}%
  \let \footnote \thanks
    {\LARGE \@title \par}%
    \vskip 1.5em%
    {\large
      \lineskip .5em%
      \begin{center}%
        \@author
      \end{center}\par}%
    \vskip 1em%
    {\large \@date}%
  \end{center}%
  \par
  \vskip 1.5em}
\makeatother

\maketitle

\begin{abstract}
\noindent
A detecção de textos gerados por modelos de linguagem de grande porte (LLMs) tornou-se uma preocupação crescente em contextos acadêmicos, educacionais e de moderação de conteúdo. Este trabalho apresenta uma primeira análise estilométrica para detecção de textos gerados por LLMs em português do Brasil. Utilizamos um corpus balanceado de 100.000 amostras (50.000 autorais, 50.000 de LLMs) extraídas de múltiplas fontes, incluindo BrWaC, ShareGPT-Portuguese e Canarim. Aplicamos 10 características estilométricas (comprimento médio de frases, relação tipo-token, entropia de caracteres, burstiness, entre outras) e realizamos testes não paramétricos (Mann-Whitney U) com correção FDR e análise de tamanho de efeito (delta de Cliff). Seis características apresentaram efeitos grandes ($|\delta| \geq 0,474$), sendo a entropia de caracteres a mais discriminante ($\delta = -0,881$). Aplicamos análise de componentes principais (PCA) e dois classificadores lineares: análise discriminante linear (LDA) e regressão logística, ambos avaliados em validação cruzada estratificada de 5 folds. A regressão logística alcançou ROC AUC de 97,03\% ($\pm 0,14\%$), enquanto a LDA obteve 94,12\% ($\pm 0,17\%$). Os resultados demonstram que métodos estatísticos clássicos são altamente eficazes para distinguir textos autorais de LLMs em português, confirmando achados anteriores em inglês e estendendo-os para outro idioma. Identificamos padrões contra-intuitivos: textos autorais são mais variáveis estruturalmente (maior burstiness e entropia), enquanto LLMs são mais diversos lexicalmente (maior TTR e proporção de hapax). Este trabalho estabelece uma base sólida para detecção estilométrica de LLMs em português e demonstra que assinaturas estilísticas humanas permanecem detectáveis através de análise estatística.
\end{abstract}

\section{Introdução}
\noindent
Neste trabalho, exploramos o uso de lógica fuzzy como método de detecção de textos gerados por modelos de linguagem de grande porte (LLMs). Para isso, construímos um classificador fuzzy baseado em métricas estilométricas - propriedades da escrita que capturam padrões linguísticos, sintáticos e semânticos. Cada métrica é associada a uma função de pertinência que expressa o grau de pertencimento de um texto a variáveis linguísticas interpretáveis, como "alta fluência" ou "baixa variação lexical".

As funções de pertinência adotadas são triangulares, determinadas por três parâmetros $(a,b,c)$, amplamente utilizadas em sistemas fuzzy por sua simplicidade algorítmica e eficiência computacional~\cite{pedrycz1994}.

O interesse em utilizar lógica fuzzy na estilometria decorre da natureza intrinsecamente gradual da linguagem. Categorias como "texto bem estruturado" ou "escrita natural" dependem de critérios de pertinência. A lógica fuzzy ocupa um espaço entre empirismo e formalidade, aproximando-se da forma como utilizamos a linguagem natural para expressar incerteza e imprecisão~\cite{klir1995}. Essa característica a torna adequada para modelar a "gradualidade" no pertencimento de um texto a uma classe (autoral ou LLM).

Ao fuzificar métricas estilométricas e combiná-las no sistema de inferência fuzzy de regras "Se ... então", é possível estimar o grau de pertencimento de um texto a cada classe.

A principal vantagem da abordagem fuzzy é a \textbf{interpretabilidade}: ao contrário de modelos de caixa-preta, os graus de pertinência podem ser inspecionados e compreendidos por humanos, revelando em que medida cada dimensão estilométrica contribui para a decisão. Além disso, o sistema fuzzy permite incorporar conhecimento linguístico especializado na definição das funções de pertinência, embora aqui seja adotada uma abordagem orientada a dados (\textit{data-driven}), determinando os parâmetros a partir de quantis das distribuições observadas.

A lógica fuzzy tem sido amplamente aplicada em processamento de linguagem natural, especialmente em análise de sentimentos~\cite{vashishtha2023} e classificação de texto~\cite{liu2024}. Trabalhos recentes também exploram sistemas fuzzy interpretativos baseados em fundamentos axiomáticos~\cite{wang2024fuzzy}, demonstrando a viabilidade de sistemas transparentes e auditáveis. Contudo, até onde sabemos, nenhum estudo anterior aplicou lógica fuzzy especificamente à detecção de textos gerados por inteligência artificial. Enquanto LLMs têm sido analisados predominantemente por métodos estatísticos ou de aprendizado profundo, este trabalho propõe a utilização de sistemas de inferência fuzzy como alternativa explicável, eficiente e de fácil interpretação.

Os resultados apresentados demonstram que classificadores fuzzy simples podem alcançar desempenho competitivo (AUC de 89\%) em comparação com abordagens estatísticas e neurais mais complexas, preservando ao mesmo tempo transparência e interpretabilidade: características essenciais para aplicações em educação, moderação de conteúdo e integridade científica.

\section{Métodos}
% Methods

\subsection{Mineração de Texto e Pré-processamento}

A mineração de texto consiste em extrair informações úteis de dados textuais não estruturados através de técnicas estatísticas e computacionais~\cite{feldman2007}. O processo envolve etapas de coleta, pré-processamento (limpeza, tokenização, normalização), extração de características numéricas e aplicação de métodos analíticos. Neste trabalho, aplicamos mineração de texto para transformar documentos em vetores de variáveis quantitativas que capturam propriedades estatísticas do estilo de escrita, permitindo análise estatística inferencial e construção de modelos de classificação.

\subsection{Conjunto de Dados}

Utilizou-se um conjunto de dados textuais balanceado em português do Brasil contendo 100.000 amostras (50.000 autorais, 50.000 de LLMs), extraídas por amostragem estratificada de um conjunto maior com 2.331.317 documentos originais provenientes de 5 fontes distintas. As fontes de texto autoral incluem: (i) BrWaC (Brazilian Web as Corpus)~\cite{brwac}, um grande conjunto web de textos brasileiros; e (ii) BoolQ~\cite{boolq}, contendo passagens de contexto para perguntas booleanas. As fontes de texto gerado por LLM incluem: (i) ShareGPT-Portuguese~\cite{sharegpt_portuguese}, conversas em português extraídas da plataforma ShareGPT; (ii) resenhas do IMDB traduzidas para português por modelos de tradução automática (classificadas como texto LLM); e (iii) o dataset Canarim~\cite{canarim}, contendo saídas geradas por LLMs.

\subsubsection{Método de Amostragem Estratificada}

A amostragem foi realizada através de \textbf{amostragem aleatória estratificada proporcional} com estratificação por fonte de origem dos textos. Este método garante representatividade de cada fonte na amostra final.

\textbf{Procedimento}:

\begin{enumerate}
    \item \textbf{Definição de estratos}: A população foi dividida em $L = 5$ estratos correspondentes às fontes:
    \begin{itemize}
        \item Estrato 1: BrWaC (textos web humanos)
        \item Estrato 2: BoolQ traduzido (textos humanos)
        \item Estrato 3: ShareGPT-Portuguese (LLM conversacional)
        \item Estrato 4: IMDB traduzido (LLM)
        \item Estrato 5: Canarim-Instruct (LLM instrucional)
    \end{itemize}

    \item \textbf{Cálculo dos tamanhos amostrais por estrato}: Para amostragem proporcional com tamanho total $n = 100.000$:
    $$n_h = n \times \frac{N_h}{N}$$
    onde $N_h$ é o tamanho populacional do estrato $h$ e $N = \sum_{h=1}^{L} N_h = 2.331.317$ é o tamanho populacional total.

    \item \textbf{Seleção aleatória simples dentro de cada estrato}: Utilizamos \texttt{numpy.random.choice} com semente fixa (42) para reprodutibilidade, sem reposição.

    \item \textbf{Combinação das amostras estratificadas}: A amostra final é a união $\bigcup_{h=1}^{L} s_h$ onde $s_h$ é a amostra do estrato $h$.
\end{enumerate}

\textbf{Vantagens da estratificação}:
\begin{itemize}
    \item \textbf{Representatividade}: Garante presença de todas as fontes proporcionalmente ao tamanho populacional
    \item \textbf{Redução de variância}: A variância da estimativa é menor que na amostragem aleatória simples quando há heterogeneidade entre estratos
    \item \textbf{Estimativas por estrato}: Permite análises separadas por fonte quando necessário
\end{itemize}

\textbf{Justificativa estatística}: A estratificação por fonte é apropriada pois diferentes fontes podem ter características textuais distintas (e.g., BrWaC contém textos web informais; Canarim contém instruções formais). A amostragem proporcional mantém a distribuição populacional original, evitando viés de seleção.

Os textos foram previamente filtrados por comprimento mínimo de 100 caracteres e máximo de 10.000 caracteres, sendo textos muito longos segmentados em fragmentos de até 10.000 caracteres sem sobreposição. A segmentação priorizou quebras naturais de texto (pontos finais, parágrafos e espaços). O balanceamento foi obtido por subamostragem da classe majoritária e sobreamostragem da classe minoritária, resultando em proporções exatamente iguais (50\%/50\%). A amostra de 100.000 documentos foi selecionada aleatoriamente com semente fixa (\texttt{seed=42}) para reprodutibilidade.

Para prevenir vazamento de dados, verificamos que os textos não apresentam agrupamentos estruturais por autor, tópico ou sessão de geração. A validação cruzada estratificada mantém o balanço de classes entre as partições, garantindo amostras independentes em conjuntos de treino e teste. Esta abordagem evita viés de avaliação documentado em estudos anteriores~\cite{kohavi1995}.

\subsection{Extração de Características Estilométricas}
\label{sec:features}

Foram extraídas 10 características estilométricas de cada documento, todas representando variáveis contínuas. A escolha dessas características baseia-se em estudos anteriores que demonstraram sua eficácia na análise de autoria \cite{stamatatos2009,stylometric_llm_detection}.

\subsubsection{Variáveis em Escala de Razão}

As nove características a seguir são mensuradas em \textbf{escala de razão}, possuindo zero absoluto e permitindo interpretação de razões:

\begin{enumerate}
    \item \textbf{Comprimento médio de frase} (\texttt{sent\_mean}): Média aritmética do número de palavras por frase. Unidade: palavras/frase. Zero representa ausência de palavras.

    \item \textbf{Desvio padrão do comprimento de frase} (\texttt{sent\_std}): Medida de dispersão absoluta do comprimento de frases. Unidade: palavras. Quantifica a variabilidade no comprimento das frases.

    \item \textbf{Coeficiente de variação do comprimento de frase} (\texttt{sent\_cv}): Razão entre desvio padrão e média ($CV = \sigma/\mu$). Estatística adimensional que normaliza a variabilidade pela tendência central, permitindo comparação entre distribuições com escalas distintas \cite{madsen2005}.

    \item \textbf{Riqueza lexical - C de Herdan} (\texttt{herdan\_c}): Medida de diversidade vocabular calculada como $C = \log(V) / \log(N)$, onde $V$ é o número de tipos (palavras distintas) e $N$ é o número de tokens (total de palavras) \cite{herdan1960}. Varia entre 0 e 1, onde valores próximos a 1 indicam maior diversidade lexical.

    \item \textbf{Proporção de pontuação} (\texttt{punct\_ratio}): Razão entre número de sinais de pontuação e total de caracteres. Adimensional, varia entre 0 e 1.

    \item \textbf{Proporção de dígitos} (\texttt{digit\_ratio}): Razão entre dígitos numéricos e total de caracteres. Adimensional, varia entre 0 e 1.

    \item \textbf{Proporção de letras maiúsculas} (\texttt{upper\_ratio}): Razão entre letras maiúsculas e total de letras. Adimensional, varia entre 0 e 1.

    \item \textbf{Proporção de palavras funcionais} (\texttt{func\_ratio}): Razão entre palavras funcionais (artigos, preposições, conjunções, pronomes) e total de palavras \cite{stamatatos2009}. Adimensional, varia entre 0 e 1. Palavras funcionais são frequentes e pouco conscientes, revelando estilo autoral.

    \item \textbf{Comprimento médio de palavra} (\texttt{word\_len\_mean}): Média do número de caracteres por palavra. Unidade: caracteres/palavra.
\end{enumerate}

\subsubsection{Variável em Escala de Intervalo}

\begin{enumerate}
    \setcounter{enumi}{9}
    \item \textbf{Variabilidade da distribuição de caracteres} (\texttt{char\_entropy}): Medida de dispersão na distribuição de frequências de caracteres, calculada pela fórmula de Shannon $H = -\sum_{c} p(c) \log_2 p(c)$ \cite{shannon1948}, onde $p(c)$ é a probabilidade de ocorrência do caractere $c$.

    Esta medida quantifica a variabilidade: alta entropia indica distribuição mais uniforme (maior dispersão); baixa entropia indica concentração (menor dispersão).

    \textbf{Justificativa estatística}: Embora originalmente uma medida da teoria da informação, a entropia funciona como \textbf{medida de dispersão análoga ao desvio padrão}, mas aplicada a distribuições de frequência categórica. A entropia é mensurada em \textbf{escala de intervalo} porque:
    \begin{itemize}
        \item Diferenças entre valores são interpretáveis (aumento de 1 bit representa dobrar a incerteza)
        \item Não possui zero absoluto natural (zero ocorre apenas com um único caractere)
        \item Razões entre valores não são estatisticamente interpretáveis
    \end{itemize}
\end{enumerate}

\subsubsection{Justificativa da Escolha das Características}

Todas as características foram selecionadas por três critérios:

\begin{enumerate}
    \item \textbf{Objetividade}: Mensuração automática e determinística, sem julgamento subjetivo.
    \item \textbf{Robustez}: Insensibilidade a pequenas variações no texto ou erros de tokenização.
    \item \textbf{Fundamentação teórica}: Suporte empírico na literatura de estilometria para distinção de autoria.
\end{enumerate}

A combinação de variáveis em escala de razão e intervalo permite aplicação de métodos estatísticos diversos. As variáveis de razão satisfazem requisitos para testes paramétricos quando distribuídas normalmente. A variável de entropia, sendo contínua em escala de intervalo, pode ser incluída em análises multivariadas que não assumem proporcionalidade (como PCA e regressão logística).

\subsection{Testes Estatísticos Não Paramétricos}
\label{sec:tests}

A escolha de métodos não paramétricos foi determinada pelas características das distribuições observadas nos dados, seguindo os critérios estabelecidos por Siegel e Castellan~\cite{siegel1988} e Hollander, Wolfe e Chicken~\cite{hollander2013}.

\subsubsection{Justificativa para Métodos Não Paramétricos}

Após análise exploratória inicial, identificamos três violações aos pressupostos de testes paramétricos:

\begin{enumerate}
    \item \textbf{Não normalidade}: Testes de Shapiro-Wilk ($\alpha = 0.05$) rejeitaram a hipótese de normalidade para 8 das 10 variáveis em ambos os grupos (humano e LLM).

    \item \textbf{Heterocedasticidade}: Teste de Levene indicou variâncias significativamente diferentes entre grupos para 6 variáveis ($p < 0.01$).

    \item \textbf{Presença de valores atípicos}: Boxplots revelaram outliers em 7 das 10 variáveis, com alguns valores extremos além de 3 desvios padrão da média.
\end{enumerate}

Dadas essas violações, métodos não paramétricos são mais apropriados pois:
\begin{itemize}
    \item Não assumem forma específica de distribuição
    \item São robustos a outliers (baseiam-se em postos, não valores brutos)
    \item Mantêm poder estatístico adequado com distribuições não normais
\end{itemize}

\subsubsection{Teste de Mann-Whitney U}

Para comparar as distribuições de cada variável entre textos humanos e LLM, utilizamos o teste de Mann-Whitney U \cite{mann1947}, também conhecido como teste de Wilcoxon para amostras independentes.

\textbf{Hipóteses}:
\begin{itemize}
    \item $H_0$: As distribuições das duas populações são idênticas
    \item $H_1$: As distribuições diferem em localização (mediana)
\end{itemize}

\textbf{Estatística do teste}:
$$U = n_1 n_2 + \frac{n_1(n_1+1)}{2} - R_1$$

onde $n_1$ e $n_2$ são os tamanhos amostrais, e $R_1$ é a soma dos postos do grupo 1.

\textbf{Interpretação}: Valores pequenos de $U$ (ou valores-$p$ menores que $\alpha$) indicam evidência contra $H_0$, sugerindo que as distribuições diferem sistematicamente.

\subsubsection{Tamanho de Efeito: Delta de Cliff}

O valor-$p$ indica apenas se há diferença estatisticamente detectável, não sua magnitude prática. Portanto, calculamos o Delta de Cliff ($\delta$) \cite{cliff1993} como medida de tamanho de efeito:

$$\delta = \frac{\#(x_i > y_j) - \#(x_i < y_j)}{n_1 \times n_2}$$

onde $x_i$ são observações do grupo 1 e $y_j$ do grupo 2.

\textbf{Interpretação} \cite{romano2006}:
\begin{itemize}
    \item $|\delta| < 0.147$: Efeito negligenciável
    \item $0.147 \leq |\delta| < 0.330$: Efeito pequeno
    \item $0.330 \leq |\delta| < 0.474$: Efeito médio
    \item $|\delta| \geq 0.474$: Efeito grande
\end{itemize}

O Delta de Cliff varia entre $-1$ e $+1$. Valores positivos indicam que o grupo 1 tende a ter valores maiores; negativos indicam o contrário.

\subsubsection{Correção para Comparações Múltiplas}

Realizamos 10 testes simultâneos (um por variável), inflando a taxa de erro Tipo I. Para controlar a \textbf{Taxa de Falsa Descoberta} (FDR - \textit{False Discovery Rate}), aplicamos o procedimento de Benjamini-Hochberg \cite{benjamini1995}:

\begin{enumerate}
    \item Ordenar os valores-$p$: $p_{(1)} \leq p_{(2)} \leq \ldots \leq p_{(10)}$
    \item Para $\alpha = 0.05$, encontrar o maior $i$ tal que:
    $$p_{(i)} \leq \frac{i}{10} \times 0.05$$
    \item Rejeitar $H_0$ para todos os testes $1, 2, \ldots, i$
\end{enumerate}

Este procedimento controla a proporção esperada de falsos positivos entre as hipóteses rejeitadas, sendo menos conservador que a correção de Bonferroni.

\subsubsection{Implementação}

Todos os testes foram implementados em Python utilizando \texttt{scipy.stats} (versão 1.11.0). Valores-$p$ foram calculados com aproximação normal para amostras grandes ($n > 20$). O Delta de Cliff foi calculado com a biblioteca \texttt{cliffs\_delta} (versão 1.0.0).

\subsection{Análise de Componentes Principais (PCA)}

Para visualizar a estrutura multivariada dos dados, aplicamos análise de componentes principais~\cite{jolliffe2002} às 10 características estilométricas. As variáveis foram previamente padronizadas (média zero, desvio padrão unitário) usando \texttt{StandardScaler} do scikit-learn~\cite{scikit-learn}. Retemos os dois primeiros componentes principais (PC1 e PC2) para visualização bidimensional. Reportamos a proporção de variância explicada por cada componente e os pesos (cargas fatoriais) de cada característica original sobre os componentes.

\subsection{Modelos de Classificação}

Avaliamos três modelos para classificação binária:

\begin{enumerate}
    \item \textbf{Análise Discriminante Linear (LDA):} um classificador generativo que assume distribuições Gaussianas multivariadas para cada classe com matrizes de covariância iguais, buscando a direção de projeção $w = S_W^{-1}(\mu_1 - \mu_2)$ que maximiza a separação entre classes~\cite{fisher1936, mclachlan2004}.

    \item \textbf{Regressão Logística:} um modelo discriminativo que estima diretamente a probabilidade posterior através da função logística $P(Y=1|X) = 1 / (1 + \exp(-(\beta_0 + \sum \beta_i x_i)))$, sem assumir normalidade das características~\cite{hosmer2013}.

    \item \textbf{Classificador Fuzzy:} um sistema baseado em regras com funções de pertinência triangulares orientadas por dados (definidas por quantis 33\%, 50\%, 66\%), agregação por média aritmética e inferência tipo Takagi-Sugeno ordem-zero. Detalhes completos em trabalho complementar sobre classificação fuzzy.
\end{enumerate}

Os modelos LDA e Regressão Logística foram treinados sobre as 10 características padronizadas (média zero, desvio padrão unitário). Para a regressão logística, utilizamos \texttt{max\_iter=1000} e sem regularização. O classificador fuzzy opera diretamente sobre as características não-padronizadas.

\subsection{Validação Cruzada e Métricas de Desempenho}

Empregamos validação cruzada estratificada com 5 partições (\texttt{StratifiedKFold}, \texttt{random\_state=42})~\cite{kohavi1995} para avaliar o desempenho dos classificadores. A estratificação garante que cada partição mantenha a proporção 50/50 de classes. Cada partição utiliza 80\% dos dados para treino (4 partições) e 20\% para teste (1 partição).

A métrica primária de avaliação é a \textbf{área sob a curva ROC (AUC)}~\cite{fawcett2006}, que resume a capacidade do modelo de discriminar entre as classes em todos os limiares de decisão. A curva ROC representa graficamente a taxa de verdadeiros positivos (sensibilidade) versus a taxa de falsos positivos (1 - especificidade) para diferentes limiares de decisão. O AUC possui interpretação probabilística: a probabilidade de que um texto LLM aleatório receba pontuação maior que um texto autoral aleatório.

Reportamos a média e o desvio padrão de AUC ao longo das 5 partições. Todas as análises foram implementadas em Python 3 utilizando as bibliotecas pandas~\cite{pandas}, NumPy~\cite{numpy}, scikit-learn~\cite{scikit-learn} e SciPy~\cite{scipy} para testes estatísticos.


\section{Resultados}
% Results

\subsection{Desempenho do Classificador Fuzzy}

A Tabela~\ref{tab:fuzzy_performance} apresenta o desempenho do classificador fuzzy proposto em validação cruzada estratificada (5 partições), juntamente com os resultados dos classificadores estatísticos (LDA e regressão logística) para comparação direta.

\begin{table}[htbp]
\centering
\caption{Comparação de desempenho entre classificador fuzzy e métodos estatísticos clássicos. Média ± desvio padrão através de 5 partições.}
\label{tab:fuzzy_performance}
\begin{tabular}{lcc}
\toprule
Modelo & ROC AUC & Precisão Média \\
\midrule
Classificador Fuzzy & $0.8934 \pm 0.0004$ & $0.8695 \pm 0.0015$ \\
LDA & $0.9412 \pm 0.0017$ & $0.9457 \pm 0.0015$ \\
Regressão Logística & $0.9703 \pm 0.0014$ & $0.9717 \pm 0.0012$ \\
\bottomrule
\end{tabular}
\end{table}

O classificador fuzzy alcança \textbf{ROC AUC de 89,34\% ($\pm 0,04\%$)}, demonstrando capacidade substancial de discriminação entre textos autorais e de LLM. Embora este desempenho seja aproximadamente 5 pontos percentuais inferior à LDA e 8 pontos percentuais inferior à regressão logística, o resultado permanece notavelmente alto, especialmente considerando a simplicidade do sistema fuzzy proposto (funções triangulares básicas com agregação por média aritmética simples).

\subsubsection{Análise de Variância e Estabilidade}

Um aspecto notável é a \textbf{estabilidade excepcional} do classificador fuzzy: o desvio padrão de AUC é de apenas $\pm 0,04\%$ ($\sigma^2 = 0,0016\%$), significativamente inferior ao de ambos os métodos estatísticos parametrizados:

\begin{itemize}
    \item LDA: $\pm 0,17\%$ ($\sigma^2 = 0,029\%$) -- \textbf{18× maior variância}
    \item Logística: $\pm 0,14\%$ ($\sigma^2 = 0,020\%$) -- \textbf{12,5× maior variância}
\end{itemize}

Esta robustez superior é atribuída à determinação de parâmetros por \textbf{quantis (estatísticas de ordem)}, que são resistentes a valores extremos (outliers) e não afetadas por assimetria distribucional~\cite{wilcox2012}. Métodos paramétricos (LDA, Regressão Logística) dependem de estimativas de média e variância, que são sensíveis a observações atípicas e violações de suposições distribucionais.

\subsubsection{Significância Estatística da Diferença de Desempenho}

A diferença de 7,9 pontos percentuais entre o classificador fuzzy (89,34\%) e a regressão logística (97,03\%) é \textbf{estatisticamente significativa}, conforme esperado dado o baixo desvio padrão de ambos os métodos. O intervalo de confiança de 95\% para a diferença é aproximadamente $[7,6\%, 8,2\%]$, indicando que a perda de desempenho é consistente e reproduzível.

Entretanto, esta diferença deve ser interpretada no contexto do \textbf{custo de oportunidade}: o classificador fuzzy sacrifica ~8\% de AUC em troca de interpretabilidade completa, robustez superior e eficiência computacional -- um trade-off justificável em aplicações onde explicabilidade é prioritária.

A Figura~\ref{fig:fuzzy_roc_comparison} apresenta a curva ROC do classificador fuzzy proposto. Observa-se que a curva permanece substancialmente acima da linha diagonal (classificador aleatório), indicando desempenho discriminatório forte.

\begin{figure}[htbp]
\centering
\includegraphics[width=0.85\textwidth]{figure_roc_fuzzy.png}
\caption{Curva ROC do classificador fuzzy proposto. A área sombreada representa $\pm 1$ desvio padrão através das 5 partições de validação cruzada. O classificador fuzzy alcança AUC de 89,34\%, demonstrando capacidade discriminatória substancial.}
\label{fig:fuzzy_roc_comparison}
\end{figure}

\begin{figure}[htbp]
\centering
\includegraphics[width=0.85\textwidth]{figure_pr_fuzzy.png}
\caption{Curva Precisão--Revocação do classificador fuzzy proposto. O classificador mantém precisão elevada em níveis moderados de revocação, com Precisão Média de 86,95\%.}
\label{fig:fuzzy_pr_comparison}
\end{figure}

\subsection{Funções de Pertinência e Interpretabilidade}

A Figura~\ref{fig:fuzzy_membership} ilustra as funções de pertinência triangulares para quatro características selecionadas: \texttt{char\_entropy}, \texttt{ttr}, \texttt{sent\_std} e \texttt{hapax\_prop}. Para cada característica, três funções fuzzy (baixo, médio, alto) são sobrepostas às distribuições empíricas de textos autorais (azul) e de LLM (laranja).

\begin{figure}[htbp]
\centering
\includegraphics[width=\textwidth]{../figure_fuzzy_membership_functions.png}
\caption{Funções de pertinência triangulares para quatro características estilométricas representativas. Linhas azul, verde e vermelha representam conjuntos fuzzy ``baixo'', ``médio'' e ``alto'', respectivamente. Histogramas sobrepostos mostram as distribuições empíricas de textos autorais (azul claro) e de LLM (laranja claro).}
\label{fig:fuzzy_membership}
\end{figure}

A visualização das funções de pertinência revela como o sistema fuzzy ``interpreta'' cada característica:

\begin{itemize}
    \item \textbf{char\_entropy:} textos autorais concentram-se na região ``alta'' (valores $> 4.5$), enquanto textos de LLM concentram-se na região ``baixa'' (valores $< 4.3$). A orientação é inversa: baixa entropia $\to$ LLM, alta entropia $\to$ autoral.

    \item \textbf{ttr:} textos de LLM apresentam TTR elevado (região ``alta'', $> 0.65$), enquanto textos autorais tendem a valores médios-baixos ($< 0.60$). A orientação é direta: baixo TTR $\to$ autoral, alto TTR $\to$ LLM.

    \item \textbf{sent\_std:} textos autorais exibem maior desvio padrão no comprimento de frases (região ``alta''), enquanto LLMs produzem textos mais uniformes (região ``baixa''). Orientação inversa: baixa variabilidade $\to$ LLM.

    \item \textbf{hapax\_prop:} similar ao TTR, LLMs produzem maior proporção de hapax legomena, concentrando-se na região ``alta''. Orientação direta: baixo hapax $\to$ autoral.
\end{itemize}

Esta transparência é a principal \textbf{vantagem} do classificador fuzzy: ao invés de produzir uma predição opaca, o sistema permite inspecionar \textit{como} e \textit{por quê} uma decisão foi tomada. Por exemplo, um texto classificado como ``80\% autoral, 20\% LLM'' pode ser analisado característica por característica para identificar quais métricas contribuíram para a decisão e em que grau.

\subsection{Análise de Custo de Oportunidade: Desempenho vs Interpretabilidade}

O classificador fuzzy oferece um \textbf{custo de oportunidade favorável} entre desempenho e interpretabilidade:

\begin{itemize}
    \item \textbf{Perda de desempenho modesta:} 7,9\% de redução em AUC comparado à regressão logística (de 97,03\% para 89,34\%).

    \item \textbf{Ganho significativo em interpretabilidade:} graus de pertinência podem ser inspecionados, visualizados e compreendidos por não-especialistas; regras fuzzy são explícitas e passíveis de auditoria.

    \item \textbf{Robustez superior:} desvio padrão 3,5× menor que LDA e 3,25× menor que regressão logística, indicando menor sensibilidade a variações nos dados.

    \item \textbf{Simplicidade computacional:} classificação requer apenas cálculo de 10 funções triangulares e uma média, sem necessidade de inversão de matrizes ou otimização iterativa.
\end{itemize}

Para aplicações onde \textit{explicabilidade} é crítica -- como educação (detectar plágio de estudantes), moderação de conteúdo (justificar decisões algorítmicas) ou integridade científica (auditar suspeitas de fraude) -- a perda modesta de desempenho pode ser amplamente compensada pela transparência do sistema fuzzy.

\subsection{Comparação com Estudos Anteriores}

Os resultados do classificador fuzzy (89,34\% AUC) são competitivos com estudos anteriores em detecção de LLMs. Por exemplo, um estudo recente usando Random Forest reportou acurácias de 81\% e 98\% em dois conjuntos de dados distintos~\cite{stylometric_llm_detection}, embora com 31 características (vs 10 neste trabalho). Nossa abordagem fuzzy, utilizando apenas 10 características simples e funções de pertinência básicas, alcança desempenho intermediário, demonstrando a viabilidade de sistemas fuzzy interpreáveis para este domínio.

Além disso, este é, ao nosso conhecimento, o \textbf{primeiro trabalho a aplicar lógica fuzzy para detecção de LLMs em português brasileiro}, contribuindo para a literatura tanto em termos metodológicos quanto linguísticos.


\section{Discussão}
% Discussion

\subsection{Interpretação dos Resultados}

Os resultados demonstram de forma conclusiva que \textbf{textos autorais e textos gerados por LLMs apresentam diferenças estilométricas substanciais em português do Brasil}. Das 10 características analisadas, 9 mostraram diferenças estatisticamente significativas com tamanhos de efeito que variam de pequeno a grande, sendo que 6 apresentaram efeitos grandes ($|\delta| \geq 0.474$). Este padrão é ainda mais robusto do que muitos estudos anteriores em língua inglesa, sugerindo que as diferenças estilísticas entre textos autorais e de LLM podem ser universais ou até mais pronunciadas em português.

A característica mais discriminante, \textbf{entropia de caracteres} ($\delta = -0.881$), revela que textos autorais apresentam distribuições de caracteres significativamente mais heterogêneas. Esta diferença pode estar relacionada a vários fatores: (i) maior diversidade de pontuação e formatação em textos autênticos (web, fóruns, redes sociais); (ii) maior variabilidade ortográfica, incluindo erros de digitação e variações dialetais; e (iii) uso mais variado de caracteres especiais, emoticons e símbolos. LLMs, treinados para gerar texto ``correto'' e bem formatado, tendem a produzir distribuições de caracteres mais uniformes e previsíveis.

A \textbf{variabilidade estrutural}, medida por \texttt{sent\_std} ($\delta = -0.790$) e \texttt{sent\_burst} ($\delta = -0.663$), também favorece fortemente textos autorais. Este resultado é consistente com a observação de que escritores exibem maior irregularidade sintática, alternando entre frases curtas e longas de forma mais natural e menos previsível. LLMs, por outro lado, tendem a gerar textos com estrutura mais regular, possivelmente devido aos mecanismos de atenção e às probabilidades de transição aprendidas durante o treinamento, que favorecem padrões consistentes.

Surpreendentemente, a \textbf{diversidade lexical} (TTR, hapax, Herdan's C) é \textit{maior} em textos de LLM. Este resultado aparentemente contra-intuitivo pode ser explicado por: (i) o treinamento em corpora extremamente vastos e diversos, expondo o modelo a vocabulário amplo; (ii) a menor tendência a repetir palavras, característica de modelos de linguagem modernos que penalizam repetição excessiva; e (iii) o fato de que textos autorais no corpus BrWaC podem incluir gêneros específicos (e.g., notícias, blogs) que naturalmente apresentam menor diversidade lexical por tratarem de tópicos especializados.

\subsection{Desempenho dos Classificadores}

O excelente desempenho dos classificadores lineares (LDA: 94,12\%, Logística: 97,03\% AUC) indica que a \textbf{separação entre as classes é aproximadamente linear} no espaço de características. Este resultado tem implicações práticas importantes: sistemas de detecção de LLMs não necessitam de arquiteturas complexas (redes neurais profundas, transformers) para alcançar alta acurácia. Métodos estatísticos clássicos, computacionalmente eficientes e facilmente interpretáveis, são suficientes.

A superioridade da regressão logística sobre LDA (~3 pontos percentuais) sugere que, embora a separação seja aproximadamente linear, as distribuições das características não são perfeitamente Gaussianas -- uma suposição central da LDA. A regressão logística, sendo um modelo discriminativo que não assume forma distribucional específica, é mais robusta a violações de normalidade, justificando sua performance superior.

A análise de componentes principais revela que PC1 (38\% de variância) representa essencialmente um eixo de ``LLM-ness'', com características de diversidade lexical (TTR, hapax) em um extremo e características de variabilidade estrutural (burstiness, entropia) no outro. Este resultado sugere que existe uma \textbf{dimensão latente fundamental} que captura a diferença entre textos autorais e de LLM, e que esta dimensão pode ser interpretada como um custo de oportunidade entre ``diversidade lexical vs variabilidade estrutural''.

\subsection{Comparação com Estudos Anteriores}

Comparando com a literatura em língua inglesa, nossos resultados são notavelmente fortes. Um estudo recente reportou acurácias de 81--98\% usando Random Forest com 31 características~\cite{stylometric_llm_detection}. Nosso trabalho alcança 97\% AUC com apenas 10 características e um modelo linear simples, sugerindo que: (i) as características estilométricas escolhidas são altamente informativas; (ii) métodos lineares podem ser tão eficazes quanto métodos ensemble para este problema; e (iii) as diferenças estilométricas em português podem ser ainda mais pronunciadas que em inglês, embora esta hipótese requeira validação com datasets paralelos.

É importante notar que a maioria dos estudos anteriores focou em inglês, deixando uma lacuna na literatura para outras línguas. Este trabalho contribui ao demonstrar que as diferenças estilométricas se generalizam para o português brasileiro, validando a universalidade (ao menos parcial) dos padrões observados e abrindo caminho para estudos multilíngues.

\subsection{Limitações}

Várias limitações devem ser reconhecidas:

\begin{enumerate}
    \item \textbf{Desbalanceamento das fontes de dados:} o corpus original era altamente desbalanceado (98\% humano, 2\% LLM), exigindo técnicas de balanceamento que podem introduzir viés. Idealmente, datasets futuros deveriam coletar amostras naturalmente balanceadas.

    \item \textbf{Diversidade de LLMs:} os textos de LLM provêm primariamente de modelos estilo ChatGPT (GPT-3.5/4). Modelos futuros ou arquiteturas distintas (e.g., Claude, Gemini, modelos especializados em português) podem apresentar padrões estilométricos diferentes, potencialmente reduzindo a acurácia dos classificadores.

    \item \textbf{Ausência de validação por tópico:} não foi possível implementar validação cruzada por tópico devido à ausência de anotações temáticas. Isto pode levar a superestimação do desempenho se tópicos específicos estiverem correlacionados com a origem do texto (humano vs LLM).

    \item \textbf{Variedade linguística limitada:} o estudo focou em português brasileiro. Português europeu e outras variantes podem apresentar padrões diferentes, limitando a generalização dos resultados.

    \item \textbf{Evolução temporal:} LLMs evoluem rapidamente. Os modelos de 2023--2024 podem gerar texto estilísticamente distinto dos modelos de 2025 em diante, potencialmente tornando os classificadores obsoletos. Estudos longitudinais são necessários para avaliar a durabilidade das características estilométricas.

    \item \textbf{Características manuais:} as 10 características foram selecionadas manualmente com base na literatura. Técnicas de seleção automática de características (e.g., LASSO, Random Forest feature importance) poderiam identificar combinações mais informativas.

    \item \textbf{Generalização entre domínios:} o estudo avalia performance em textos genéricos de múltiplas fontes, mas não testa explicitamente generalização cross-domain. Evidências da literatura~\cite{brennan2016} demonstram que características estilométricas podem degradar significativamente quando treinadas em um domínio (e.g., acadêmico) e testadas em outro (e.g., redes sociais). Avaliação futura deveria incluir testes em domínios específicos (notícias, literatura, código, conversas) para validar robustez.

    \item \textbf{Limitações do Type-Token Ratio:} a métrica TTR tem sido criticada desde 1987~\cite{richards1987} por dependência do comprimento do texto. Alternativas como MTLD (Measure of Textual Lexical Diversity)~\cite{mccarthy2010} oferecem medidas invariantes ao tamanho e poderiam fortalecer a análise.
\end{enumerate}

\subsection{Implicações Práticas}

Os resultados têm implicações diretas para várias aplicações:

\begin{itemize}
    \item \textbf{Educação:} sistemas de detecção de plágio podem incorporar características estilométricas para identificar trabalhos gerados por IA, auxiliando educadores a manter a integridade acadêmica.

    \item \textbf{Moderação de conteúdo:} plataformas online podem usar classificadores estilométricos para detectar spam, desinformação ou conteúdo gerado automaticamente em massa.

    \item \textbf{Integridade científica:} editores e revisores podem aplicar análise estilométrica para identificar manuscritos suspeitos gerados (total ou parcialmente) por LLMs, especialmente em áreas onde a originalidade é crítica.

    \item \textbf{Forense digital:} análise forense de textos pode beneficiar-se de métodos estilométricos para atribuição de autoria ou detecção de manipulação.
\end{itemize}

Entretanto, é importante ressaltar que \textbf{classificadores estilométricos não devem ser usados de forma punitiva sem investigação adicional}. Falsos positivos podem prejudicar indivíduos inocentes, e a detecção automática deve ser vista como uma ferramenta de triagem, não como veredicto final.

\subsection{Direções Futuras}

Trabalhos futuros podem explorar:

\begin{enumerate}
    \item \textbf{Estudos multilíngues:} aplicar a mesma metodologia a outras línguas (espanhol, francês, alemão, etc.) para avaliar a universalidade dos padrões estilométricos.

    \item \textbf{Análise longitudinal:} coletar dados de múltiplas gerações de LLMs e avaliar como as características estilométricas evoluem ao longo do tempo.

    \item \textbf{Detecção em domínios específicos:} avaliar desempenho em gêneros textuais específicos (acadêmico, jornalístico, literário, código) onde LLMs podem comportar-se diferentemente.

    \item \textbf{Textos híbridos:} desenvolver métodos para detectar textos parcialmente editados por humanos após geração por LLM (cenário comum em uso real).

    \item \textbf{Características neurais:} combinar características estilométricas clássicas com embeddings contextuais (BERT, GPT) para classificação híbrida.

    \item \textbf{Explicabilidade:} desenvolver visualizações interativas que permitam usuários finais compreender \textit{por que} um texto foi classificado como humano ou LLM.
\end{enumerate}


\section{Conclusão}
\noindent
Este trabalho apresentou um \textbf{classificador baseado em lógica fuzzy para detecção de textos gerados por modelos de linguagem de grande porte (LLMs) em português do Brasil}. Utilizando funções de pertinência triangulares simples e um sistema de inferência baseado em média, alcançamos um desempenho de 89,34% em ROC AUC - resultado que demonstra a viabilidade de sistemas fuzzy interpretáveis para esta tarefa.

As principais contribuições podem ser resumidas da seguinte forma:

\begin{enumerate}
\item \textbf{Aplicação de lógica fuzzy na detecção de textos gerados por LLMs:} o trabalho amplia a literatura de estilometria e detecção de texto automático ao introduzir uma abordagem alternativa aos métodos estatísticos e de aprendizado de máquina convencionais.

\item \textbf{Sistema orientado a dados e livre de conhecimento especialista:} os parâmetros das funções de pertinência foram determinados a partir de quantis das distribuições observadas, eliminando a necessidade de ajustes manuais e tornando o método escalável, objetivo e reproduzível.

\item \textbf{Quantificação do custo de oportunidade entre interpretabilidade e desempenho:} foi demonstrado que o custo da explicabilidade é modesto (cerca de 8\% de perda em AUC), o que torna o modelo adequado para cenários em que transparência e auditabilidade são prioritárias.

\item \textbf{Alta robustez:} o classificador fuzzy apresentou variância 3–4× menor que a observada em métodos comparativos, indicando maior estabilidade sob diferentes amostras e condições de teste.

\item \textbf{Visualizações linguísticas interpretáveis:} as funções de pertinência e seus graus de ativação oferecem uma compreensão direta de como cada métrica estilométrica contribui para distinguir textos autorais de textos produzidos por LLMs.


\end{enumerate}

O desempenho obtido evidencia que sistemas fuzzy podem competir com abordagens mais complexas, preservando vantagens cruciais de interpretabilidade e transparência. Essa característica é especialmente relevante em domínios sensíveis - como educação, moderação de conteúdo e integridade científica - onde decisões algorítmicas precisam ser explicáveis e justificáveis.

As principais limitações do presente estudo incluem a simplicidade da modelagem (funções triangulares e agregação média) e a ausência de testes em contextos temáticos especializados. Pesquisas futuras podem explorar sistemas fuzzy mais sofisticados, com operadores de agregação alternativos, aprendizado automático de parâmetros e regras multidimensionais, além de validação em múltiplos domínios linguísticos.

Em síntese, \textbf{a lógica fuzzy representa um caminho promissor para a detecção interpretável de textos gerados por LLMs}, servindo como ponte entre modelos estatísticos e métodos baseados em aprendizado profundo. À medida que os sistemas de IA tornam-se mais difundidos e suas decisões mais impactantes, abordagens que conciliam desempenho e explicabilidade - como a apresentada neste trabalho - serão fundamentais para o desenvolvimento ético e transparente da inteligência artificial.

\bibliography{refs}

\end{document}